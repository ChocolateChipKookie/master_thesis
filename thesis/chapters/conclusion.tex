\chapter{Conclusion}

All the results showcased in this thesis should be taken with a grain of salt, 
both in the negative, but also in the positive direction. Namely, the training was 
done with less time and resources than the authors of the referenced papers had.
So all the trained models probably produce worse results than their original counterparts.

\begin{figure}[!ht]
	\centering
	\begin{subfigure}{.4\textwidth}
		\centering
		\includegraphics[width=.9\linewidth]{img/fail/undercolored}
        \caption{Unsaturated colorization}
	\end{subfigure}
    \begin{subfigure}{.4\textwidth}
		\centering
		\includegraphics[width=.9\linewidth]{img/fail/nashville}
        \caption{Overconfident colorization}
    \end{subfigure}
    \caption{Colorization failure cases}
    \label{fig:fail}
\end{figure}

On the other hand, in the galleries of the trained models only good examples were 
showcased. Generally, good colorization with these models is achieved on roughly 50\%
of the input images. While in most failure cases the output image is is either uncolored 
or unsaturated, sometimes the colorizer is overconfident in a bad colorization as seen 
in figure \ref{fig:fail}.

In general two improvements to the training process could be made for all models.
The first improvement has to do with the dataset. A better dataset could be 
constructed specifically for this task. It should not contain any monochrome images, 
and desaturated/oversaturated images should be filtered out as well. 

The second improvement has to do with the preprocessing of the training images. 
As colorization models usually find the most use in colorizing old photographs, 
the models should be immune to the slight noise/graininess of the old images. 
The models trained as a part of this thesis performed worse on old and grainy images
as they were trained mostly on images taken with digital cameras that had almost no 
noise. This could simply be fixed by artificially adding noise while preprocessing
the input training images.

\section{Result Comparison}

\begin{figure}[!ht]
	\centering
	\begin{subfigure}{.23\textwidth}
		\centering
		\includegraphics[width=\linewidth]{img/comparison/bw/flag}
	\end{subfigure}
    \begin{subfigure}{.23\textwidth}
		\centering
		\includegraphics[width=\linewidth]{img/comparison/colorful/flag}
    \end{subfigure}
    \begin{subfigure}{.23\textwidth}
		\centering
		\includegraphics[width=\linewidth]{img/comparison/gan/flag}
    \end{subfigure}
    \begin{subfigure}{.23\textwidth}
		\centering
		\includegraphics[width=\linewidth]{img/comparison/ideep/flag}
    \end{subfigure}
	\begin{subfigure}{.23\textwidth}
		\centering
		\includegraphics[width=\linewidth]{img/comparison/bw/jazz}
	\end{subfigure}
    \begin{subfigure}{.23\textwidth}
		\centering
		\includegraphics[width=\linewidth]{img/comparison/colorful/jazz}
    \end{subfigure}
    \begin{subfigure}{.23\textwidth}
		\centering
		\includegraphics[width=\linewidth]{img/comparison/gan/jazz}
    \end{subfigure}
    \begin{subfigure}{.23\textwidth}
		\centering
		\includegraphics[width=\linewidth]{img/comparison/ideep/jazz}
    \end{subfigure}
	\begin{subfigure}{.23\textwidth}
		\centering
		\includegraphics[width=\linewidth]{img/comparison/bw/lumber}
	\end{subfigure}
    \begin{subfigure}{.23\textwidth}
		\centering
		\includegraphics[width=\linewidth]{img/comparison/colorful/lumber}
    \end{subfigure}
    \begin{subfigure}{.23\textwidth}
		\centering
		\includegraphics[width=\linewidth]{img/comparison/gan/lumber}
    \end{subfigure}
    \begin{subfigure}{.23\textwidth}
		\centering
		\includegraphics[width=\linewidth]{img/comparison/ideep/lumber}
    \end{subfigure}
	\begin{subfigure}{.23\textwidth}
		\centering
		\includegraphics[width=\linewidth]{img/comparison/bw/mlk}
	\end{subfigure}
    \begin{subfigure}{.23\textwidth}
		\centering
		\includegraphics[width=\linewidth]{img/comparison/colorful/mlk}
    \end{subfigure}
    \begin{subfigure}{.23\textwidth}
		\centering
		\includegraphics[width=\linewidth]{img/comparison/gan/mlk}
    \end{subfigure}
    \begin{subfigure}{.23\textwidth}
		\centering
		\includegraphics[width=\linewidth]{img/comparison/ideep/mlk}
    \end{subfigure}
	\begin{subfigure}{.23\textwidth}
		\centering
		\includegraphics[width=\linewidth]{img/comparison/bw/nashville}
	\end{subfigure}
    \begin{subfigure}{.23\textwidth}
		\centering
		\includegraphics[width=\linewidth]{img/comparison/colorful/nashville}
    \end{subfigure}
    \begin{subfigure}{.23\textwidth}
		\centering
		\includegraphics[width=\linewidth]{img/comparison/gan/nashville}
    \end{subfigure}
    \begin{subfigure}{.23\textwidth}
		\centering
		\includegraphics[width=\linewidth]{img/comparison/ideep/nashville}
    \end{subfigure}
	\begin{subfigure}{.23\textwidth}
		\centering
		\includegraphics[width=\linewidth]{img/comparison/bw/whitman}
		\caption{Original grayscale image}
	\end{subfigure}
    \begin{subfigure}{.23\textwidth}
		\centering
		\includegraphics[width=\linewidth]{img/comparison/colorful/whitman}
		\caption{Colorful Image Colorization}
    \end{subfigure}
    \begin{subfigure}{.23\textwidth}
		\centering
		\includegraphics[width=\linewidth]{img/comparison/gan/whitman}
		\caption{Generative Adversarial Network}
    \end{subfigure}
    \begin{subfigure}{.23\textwidth}
		\centering
		\includegraphics[width=\linewidth]{img/comparison/ideep/whitman}
		\caption{Interactive Deep Colorization}
    \end{subfigure}
\end{figure}