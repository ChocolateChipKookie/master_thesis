\begin{comment}
    Use cases
    Multimodality

    Neural networks:
        Basics
        CNN
        U-NET

    Dataset

    Implementations:
        Colorful image colorization
        GAN/pix2pix
        IDeep
\end{comment}

\chapter{Methods}

The task of automatic colorization can be defined as an image-to-image 
translation problem \citep{pang2021imagetoimage}, where for a monochrome 
input image we have to propose a plausible colored output. Although 
grayscale is a subset of monochrome (meaning \textit{having one color}), 
the two terms will be used interchangeably throughout this thesis. 
The defining feature of the input images is that they only have one channel 
representing the perceptual lightness of the given image. That means that 
each pixel only has a numeric value, usually in the 8-bit range, 
expressing how light (or dark) it is. 

\begin{figure}[H]
    \centering
    \includegraphics[width=0.5\textwidth]{img/alim_khan.jpg}
    \caption{
    Colored image of Alim Khan taken by Sergey Prokudin-Gorsky 
    in 1911 using the three-image color photography method on the left,
    and the green, red, and blue (top to bottom) filter negatives
    (shown as positives) on the right.}
    \label{fig:alim_khan}
\end{figure}

On the other hand, if we want to render a colored photograph we need at 
least 3 channels to do, meaning that for each pixel we want to draw, 
we need at least 3 values associated with it. The specific way of mapping 
the three values to a color is called a color space. An example of a color
space is the RGB color model in which each of the 3 channels depicts the 
amount of \textbf{R}ed, \textbf{G}reen, or \textbf{B}lue color present,
as shown in figure \ref{fig:alim_khan}. 

That said, the task of colorization can be summed up as
    \[\mathbf{Y} = \mathcal{F}(\mathbf{X})\]
where $\mathbf{X} \in \mathbb{R}^{H \times W \times 1}$ is the input 
and $\mathbf{Y} \in \mathbb{R}^{H \times W \times 3}$ is the colored output.
$\mathcal{F}$ is the function that maps the domain $\mathbf{X}$ to the codomain 
$\mathbf{Y}$, which in our case represents the colorizer.


\section{Color space}

Even though automatic colorization can be accomplished using any color space, 
as with any task some tools are better for the job than others. Although the 
RGB color space would be the easiest to use, as it is the default option for 
many tools, it doesn't give us an advantage when colorizing. On the other hand, 
the CIELAB color space has a few useful properties which make it convenient 
for the task. The CIELAB (or L*a*b*) color space was defined by the 
International Commission on Illumination in 1976 as an attempt to create a 
perceptually uniform color space \citep{icc2004cielab}. 

In perceptually uniform color spaces, a numeric change in any direction 
would result in a similar perceived color change. Although L*a*b* is not truly 
perceptually uniform, it is nevertheless useful for that purpose. Perceptual 
uniformity is not necessary while colorizing, but it is practical. As small 
changes in value result in small color changes.


- What L*a*b* mean
- Plots of lab colorspace
- Not as effective for image compression as only a third of the box is used
- Quintization errors with ints, best to use 32 bit floats to repersent the colorspace
- Covers the whole visible color spectrum
- Created as perceptually coherent color space
- Not entirely successful at it, but quite successful
- Why use LAB colorspace
- Good for this task because of:
    - Perceptual distance close to numerical distance (good at assesing what humans would percieve color differences)
    - Input diemnsion is grayscale which matches the L channel of Lab colorspace, we only need to predict 2 channels a* and b*

As we are working with images the choice of colorspace is quite important 
Covers the while human visible spectrum
Perceptually close colors are 

thats why rather than using the rgb colorspace ...  La*b*


- 



\section{Task definition}



- That said
- The task of colorization is:
    get ab color channels from the luminance value of the photo
    image-to-image translation problem where the domain is the grayscale L channel and codomain are the ab color channels 
    Trying to imagine the probable values for the a and b (color) channels
    of the CIELab colorspace givent the luminance values of a grayscale image
    Colored images have 3 channels, one channel is already available, so the colorizer has to 
    hallucinate



- It is impossible to perfectly recover the original image as too much information has been lost due to the colorspace
- The best the model can do is to offer a plausible colorization
- Trying to imagine 2 color dimensions of the lab color space

- Image colorization definition

- The baseline knowledge that is needed to understand
- Approaches to the problem
- Dataset and dataset preparation (possible improvements like adding noise)
- describing 


\subsection{Multimodality}

\section{Dataset}
\label{sec:dataset}

- The model is only as good as the dataset


\begin{comment}

    
    
    \section{Multimodality}
    \label{sec:multimodality}
    
    Multimodalnost
    
    - Multimodality:
        Roses are red, violets are grey?
        There is more to it
    
    
    The task of image colorization can be defined as an image-to-image translation problem, where 
    given a grayscale input image, the colorizer has to suggest a plausible colored version 
    of the same image. The input to the colorizer is a 2-dimensional array 
    where the intensity of each pixel is a one-dimensional 
    value (usually in the 8-bit range), whereas the output, is a colored image that 
    requires at least three channels to cover the whole color spectrum. 
    
    The process of colorization can be written down as:
        \[\mathbf{Y} = \mathcal{F}(\mathbf{X})\]
    where $\mathbf{X} \in \mathbb{R}^{H \times W \times 1}$ is the input grayscale image, 
    $\mathbf{Y} \in \mathbb{R}^{H \times W \times 3}$ is the colored output and
    $\mathcal{F}$ is the function that maps $\mathbf{X} \rightarrow \mathbf{Y}$. 
    In the case of colorization $\mathcal{F}$ represents the colorizer.
    
    
    
    \section{Use Cases}
    
    The primary use case of colorization is to add color to old monochrome images taken
    in an era where colored photography was not yet a viable option. Although people 
    can imagine how the photographed scene looked like when the image was taken, seeing 
    a plausible colorization gives the viewer another perspective to the original image.
    Automatic colorization might also be useful in the field of image\citep{fatima2021image} 
    and video \citep{pan2019video} compression as saving only the luminance channel 
    of an image or video (and some color information) reduces the file size with a potentially
    low impact on the image quality. 
\end{comment}
