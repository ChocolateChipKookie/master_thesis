\chapter{Introduction}
\label{ch:introduction}
\begin{comment}
    The introduction to the thesis

    Colorization definition

    Topics to cover:
        What is image colorization
        Why is it useful
        What can it be used for
        Description of the task

        Not sure about previous work, if there is enough text, it might be its own chapter
\end{comment}

Ever since the advent of photography in the early 19th century people had
the urge to add color to photographs. Not long after the invention of the 
daguerreotype by Louis Daguerre in 1839, the Swiss painter and later photographer 
Johann Baptist Isenring produced the first colored photographic image by 
hand-applying pigments to the surface of a monochrome photograph \citep{henisch1996photograph}. 
Although first experiments with color photography date back to the middle of the 19th 
century hand-coloring monochrome photographs, as seen in figure \ref{fig:hand_colored}, 
was the easiest method to produce full-color photographic images until the arrival 
of color film in the mid-20th century. With the beginning of the digital age and 
the increase of computational power the fascination with coloring images reemerged, 
but this time the problem was tackled from a computational perspective.

\begin{figure}[H]
    \centering
    \includegraphics[width=0.54\textwidth]{img/hand_colored.jpg}
    \caption{Hand-colored photograph by Stillfried \& Andersen (made the period between 1862 and 1885)}
    \label{fig:hand_colored}
\end{figure}

At first sight, the task of automatic colorization might look impossible to accomplish, 
as information is lost when taking a monochrome photograph. If our goal is to 
perfectly colorize an image that assumption is valid most of the time because we 
encounter the problem of multimodality that is described in section 
\ref{sec:multimodality}. If, on the other hand, we are satisfied with a plausible 
colorization, the task becomes achievable as there is still a lot of semantic 
information ingrained in the grayscale image. 

\begin{figure}[ht!]
	\centering
	\begin{subfigure}{.32\textwidth}
		\centering
		\includegraphics[width=.9\linewidth]{img/colorization/house_original.jpg}
        \caption{Original}
		\label{fig:stone_house:original}
	\end{subfigure}
    \begin{subfigure}{.32\textwidth}
		\centering
		\includegraphics[width=.9\linewidth]{img/colorization/house_grayscale.jpg}
        \caption{Grayscale}
        \label{fig:stone_house:grayscale}
    \end{subfigure}
	\begin{subfigure}{.32\textwidth}
		\centering
		\includegraphics[width=.9\linewidth]{img/colorization/house_colorized.jpg}
        \caption{Colorized}
        \label{fig:stone_house:colorized}
    \end{subfigure}
    \caption{Colorization example on an image of a stone house}
	\label{fig:stone_house}
\end{figure}

For example, if our colorizer is given the grayscale image in figure 
\ref{fig:stone_house} it can assume that the vegetation surrounding the 
house is green, while the stone walls are probably some shade of grey. 
All that information is available no matter that there is no color 
in the image, but the colorizer should be able to deduce a likely color from the 
shapes and textures present. 

That said, in this thesis we will explore the topic of automatic image 
colorization. Firstly the task of image colorization will be described as well 
other notable details important for the understanding of that topic. After the
basic are covered 

\begin{comment}

\textbf{TODO!}
% TODO: OVO SE TREBA NA KRAJU NAPISATI


- Technicalities 
- The task of colorization
- Different approaches to image colorization
- Dataset
- Problems with colorization
- Different metrics for assesing the models
- Ground up explinations of the working and training of different models
\end{comment}