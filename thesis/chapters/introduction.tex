\chapter{Introduction}

\begin{comment}
    The introduction to the thesis

    Colorization definition

    Topics to cover:
        What is image colorization
        Why is it useful
        What can it be used for
        Description of the task

        Not sure about previous work, if there is enough text, it might be its own chapter

\end{comment}

Ever since the advent of photography in the early 19th century people had
the urge to add color to photographs. Not long after the invention of the 
daguerreotype by Louis Daguerre in 1839, the Swiss painter and later photographer 
Johann Baptist Isenring produced the first colored photographic image by 
hand-applying pigments to the surface of a monochrome photograph. Although 
first experiments with color photography date back to the middle of the 19th 
century, hand-coloring monochrome photographs was the easiest method to 
produce full-color photographic images until the arrival of color film in the mid-20th century.

- The fascination with hand colorization has died down
- Reemerged in the early 2000 (statistical models, descriptions...)
- Significant progress with convolutional networks
- Modern

\section{Image Colorization}

- Task description
- Given a monochrome image ...
- First attempts with 

\section{Previous work}



\section{Use Cases}

