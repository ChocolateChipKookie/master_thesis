\chapter{Introduction}

\begin{comment}
    The introduction to the thesis

    Colorization definition

    Topics to cover:
        What is image colorization
        Why is it useful
        What can it be used for
        Description of the task

        Not sure about previous work, if there is enough text, it might be its own chapter
\end{comment}

Ever since the advent of photography in the early 19th century people had
the urge to add color to photographs. Not long after the invention of the 
daguerreotype by Louis Daguerre in 1839, the Swiss painter and later photographer 
Johann Baptist Isenring produced the first colored photographic image by 
hand-applying pigments to the surface of a monochrome photograph \citep{henisch1996photograph}. 
Although first experiments with color photography date back to the middle of the 19th 
century hand-coloring monochrome photographs, as seen in figure \ref{fig:hand_colored}, 
was the easiest method to produce full-color photographic images until the arrival 
of color film in the mid-20th century. With the beginning of the digital age and 
the increase of computational power the fascination with coloring images reemerged, 
but this time the problem was tackled from a computational perspective.

\begin{figure}[H]
    \centering
    \includegraphics[width=0.55\textwidth]{img/hand_colored.jpg}
    \caption{Hand-colored photograph by Stillfried \& Andersen (made the period between 1862 and 1885)}
    \label{fig:hand_colored}
\end{figure}

\section{Task description}

At first sight, the task of automatic colorization might look impossible to accomplish, 
as information is lost when taking a monochrome photograph. If our goal is to 
perfectly colorize an image that assumption is valid most of the time because we 
encounter the problem of multimodality that is described in section 
\ref{sec:multimodality}. If, on the other hand, we are satisfied with a plausible 
colorization, the task becomes achievable as there is still a lot of semantic 
information ingrained in the grayscale image. 

\begin{figure}[h!]
	\centering
	\begin{subfigure}{.32\textwidth}
		\centering
		\includegraphics[width=.9\linewidth]{img/house_original.jpg}
        \caption{Original}
		\label{fig:stone_house:original}
	\end{subfigure}
    \begin{subfigure}{.32\textwidth}
		\centering
		\includegraphics[width=.9\linewidth]{img/house_grayscale.jpg}
        \caption{Grayscale}
        \label{fig:stone_house:grayscale}
    \end{subfigure}
	\begin{subfigure}{.32\textwidth}
		\centering
		\includegraphics[width=.9\linewidth]{img/house_colorized.jpg}
        \caption{Colorized}
        \label{fig:stone_house:colorized}
    \end{subfigure}
    \caption{Colorization example on an image of a stone house}
	\label{fig:stone_house}
\end{figure}

For example, if our colorizer is given the grayscale image in figure 
\ref{fig:stone_house} it can assume that the vegetation surrounding the 
house is green, while the stone walls are probably some shade of grey. 
All that information is available no matter that there is no color 
in the image, but the colorizer should be able to deduce a likely color from the 
shapes and textures present. 

That said the task of colorization can be described as a image-to-image translation problem

- That said
- The task of colorization is:
    get ab color channels from the luminance value of the photo
    image-to-image translation problem where the domain is the grayscale L channel and codomain are the ab color channels 
    Trying to imagine the probable values for the a and b (color) channels
    of the CIELab colorspace givent the luminance values of a grayscale image



- It is impossible to perfectly recover the original image as too much information has been lost due to the colorspace
- The best the model can do is to offer a plausible colorization
- Trying to imagine 2 color dimensions of the lab color space


\section{Multimodality}
\label{sec:multimodality}

Multimodalnost

- Multimodality:
    Roses are red, violets are grey?
    There is more to it

\begin{comment}

The task of image colorization can be defined as an image-to-image translation problem, where 
given a grayscale input image, the colorizer has to suggest a plausible colored version 
of the same image. The input to the colorizer is a 2-dimensional array 
where the intensity of each pixel is a one-dimensional 
value (usually in the 8-bit range), whereas the output, is a colored image that 
requires at least three channels to cover the whole color spectrum. 

The process of colorization can be written down as:
    \[\mathbf{Y} = \mathcal{F}(\mathbf{X})\]
where $\mathbf{X} \in \mathbb{R}^{H \times W \times 1}$ is the input grayscale image, 
$\mathbf{Y} \in \mathbb{R}^{H \times W \times 3}$ is the colored output and
$\mathcal{F}$ is the function that maps $\mathbf{X} \rightarrow \mathbf{Y}$. 
In the case of colorization $\mathcal{F}$ represents the colorizer.

\end{comment}


\section{Use Cases}

The primary use case of colorization is to add color to old monochrome images taken
in an era where colored photography was not yet a viable option. Although people 
can imagine how the photographed scene looked like when the image was taken, seeing 
a plausible colorization gives the viewer another perspective to the original image.
Automatic colorization might also be useful in the field of image\citep{fatima2021image} 
and video \citep{pan2019video} compression as saving only the luminance channel 
of an image or video (and some color information) reduces the file size with a potentially
low impact on the image quality. 


\section{Related work}

- Beginning statistical models
- Convolutional networks
- Generative Adversarial networks
- In the last few years novel approaches: Transformers, cINN
- Specialized approaches for different use cases