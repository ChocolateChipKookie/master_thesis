\chapter{Related work}

The field of automatic image colorization traces its roots to the early 2000s
when the first attempts at computer-assisted colorization were made. The paper
by Reinhard et al. \citep{reinhard2001colorization} suggested a color transfer method
based on color matching. Not long after, inspired by the paper by Reinhard et al., 
Welsh et al. proposed a color transfer method that used texture synthesis as the main
colorization mechanism \citep{welsh2002colorization}. 

In 2004 a new approach to colorization was described by Levin et al. 
\citep{levin2004colorization}. In their paper, the authors described a method of 
colorization with colored strokes as user inputs. With less than a dozen user inputs the
algorithm was able to produce plausible images. Until the end of the decade, several 
new methods of colorization were proposed, but they just refined the already 
existing algorithms \citep{luan2007colorization}\citep{charpiat2008colorization}.

A common feature among all the mentioned colorization algorithms is that 
they were data-driven, statistical methods. In the 2010s, with the rapid development 
of neural networks (and CNNs in particular) that changed, and the focus shifted
to parametric methods trained on large datasets 
\citep{cheng2015colorization}\citep{deshpande2015colorization}. The user-guided methods
no longer needed strokes of color, as points of color became sufficient to nudge 
the models in the right direction \citep{zhang2017ideep}.

Although user-guided colorization plateaued with the aforementioned CNN 
models, fully automatic colorization has seen significant progress in the 
last couple of years. Novel parametric approaches like the 
\textit{PixColor: Recursive Pixel Colorization} \citep{guadarrama2017colorization},
generative adversarial networks based models \citep{nazeri2018gan}, and the \textit{Colorization 
Transformer} \citep{kumar2021colorization}, all attempt to solve the unsolvable by using creative 
ideas.